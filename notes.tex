\documentclass[12pt]{article}

\usepackage{fouriernc}
\usepackage{amssymb}
\usepackage{amsmath}
\usepackage{amsfonts}
\usepackage[utf8]{inputenc}
\usepackage[T1]{fontenc}
\usepackage[margin=1in]{geometry}
\usepackage{graphicx}
\usepackage{multicol}

\graphicspath{ {./images/} }

\newcommand{\curly}[1]{\left\{      #1 \right\}     }
\newcommand{\round}[1]{\left(       #1 \right)      }
\newcommand{\hard} [1]{\left[       #1 \right]      }
\newcommand{\abs}  [1]{\left|       #1 \right|      }
\newcommand{\floor}[1]{\left\lfloor #1 \right\rfloor}
\newcommand{\ceil} [1]{\left\lceil  #1 \right\rceil }
\newcommand{\R}    [0]{\mathbb{R}                   }
\newcommand{\Z}    [0]{\mathbb{Z}                   }
\newcommand{\N}    [0]{\mathbb{N}                   }

\setlength{\parindent}{0in}

\begin{document}
\begin{multicols}{2}

\section*{Divergence tests}

\subsection*{Simple}
\begin{align*}
    \lim_{n\to\infty} a_n \neq 0 \implies \sum a_n = \infty
\end{align*}

\subsection*{Integral}
\begin{align*}
    \int f(x)\ dx = \infty &\implies \sum f(n) = \infty \\
    \int f(x)\ dx < \infty &\implies \sum f(n) < \infty
\end{align*}

\subsection*{Comparison}
\begin{align*}
    b_n > a_n,\ \sum a_n = \infty &\implies \sum b_n = \infty \\
    b_n < a_n,\ \sum a_n < \infty &\implies \sum b_n < \infty
\end{align*}

\subsection*{Ratio}
\begin{align*}
    \lim_{n\to\infty} \frac{a_{n+1}}{a_n} < 1 &\implies \sum a_n < \infty\\
    \lim_{n\to\infty} \frac{a_{n+1}}{a_n} > 1 &\implies \sum a_n = \infty
\end{align*}

\subsection*{Root}
\begin{align*}
    \lim_{n\to\infty} \sqrt[n]{a_n} < 1 &\implies \sum a_n < \infty \\
    \lim_{n\to\infty} \sqrt[n]{a_n} > 1 &\implies \sum a_n = \infty
\end{align*}

\subsection*{Limit comparison}
Let $L = \lim_{n\to\infty} \frac{a_n}{b_n}$.
\begin{enumerate}
    \item If $L=0$ and $\sum b_n < \infty$ then $\sum a_n < \infty$.
    \item If $0<L<\infty$ then $\sum a_n$ and $\sum b_n$ behave the same.
    \item If $L=\infty$ and $\sum b_n = \infty$ then $\sum a_n = \infty$.
\end{enumerate}

\section*{Power series}
\begin{align*}
    \forall \abs{x} < 1,\quad \sum_{n=0}^\infty x^n = \frac{1}{1-x}
\end{align*}

Remember that $b_n$ refers to the coefficient of $x_n$, not necessarily the coefficient of the $n$th term.

Given $\sum_{n=0}^\infty b_n (x-c)^n$, we have absolute convergence when
\begin{align*}
    \abs{x-c} < \lim_{n\to\infty} \abs{\frac{b_{n+1}}{b_n}}
\end{align*}

\subsection*{Taylor series}
Centered around $x=c$,
\begin{align*}
    f(x) &= \sum_{k=0}^\infty \frac{f^{(n)}(c) (x-c)^k}{k!}
\end{align*}

\end{multicols}
\end{document}
